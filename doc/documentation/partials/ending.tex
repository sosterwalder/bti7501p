\section{Schlusswort}
\label{chap:ending}
Voronoi-Diagramme zeigen auf eine eindrucksvolle Weise, wie sich Ebenen in logarithmischer Zeit in Regionen unterteilen lassen, welche dann z.B. mittels der Delaunay-Triangulation, dank deren Dualität, in linearer Zeit trianguliert werden können. Dies eröffnet sehr interessante Möglichkeiten, gerade zum Beispiel wenn wenig Speicherplatz zur Verfügung steht. So können zum Beispiel, mit absehbarem Aufwand, beliebige dreidimensionale, triangulierte Modelle aus gegebenen oder auch zufällig gewählten Punkten zur Darstellung von Computergrafiken erzeugt werden.

Weiter lassen die beiden Methoden (unter entsprechender Erweiterung, z.B. durch Heuristiken) das Lösen bzw. Annähern von NP-schweren Problemen, wie zum Beispiel das des Handlungsreisenden, zu~\parencite{schmitting2007}.
