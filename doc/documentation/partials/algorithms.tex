\section{Algorithmen}
\label{sec:algorithms}
In der aktuellen Version der Applikation wurde der in der Spieltheorie weit verbreitete Ansatz des Minimax-Verfahrens inklusive Alpha-Beta-Pruning gewählt.

Wie in~\cite{eckerle2014spieltheorie} erwähnt, ist der Algorithmus wie folgt aufgebaut:\\

\begin{lstlisting}
Begin
    If (X ist ein Terminalknoten) then 
        return e(x);

    Elseif (X ist ein MAX-Knoten) then

        Generiere die Menge S(X) aller Kindknoten von X;

        V = -1;
        Foreach Y aus S(X) do v = max(v, Minimax(Y));
        Return v;

    Else (X ist ein MIN-Knoten) then
        Generiere die Menge S(X) aller Kindknoten von X;
        V = 1;
        Foreach Y aus S(X) do v= min(v, Minimax(Y));
        Return v;

    Endif;

End
\end{lstlisting}

Um die Laufzeit bei optimal gewählter Heuristik auf $O(b^{d/2})$ optimieren zu können, wurde zusätzlich das Alpha-Beta-Pruning gewählt~\parencite{eckerle2014spieltheorie}:\\

\lstinputlisting[language=Python, firstline=63, lastline=72]{../../algorithms/mini_max_alpha_beta.py}

bzw.:\\

\lstinputlisting[language=Python, firstline=88, lastline=102]{../../algorithms/mini_max_alpha_beta.py}
